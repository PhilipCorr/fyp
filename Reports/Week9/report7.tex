\documentclass[]{weekly-report}
\usepackage{hyperref}
 
%%%%%%%%%%%%%%%%%%%%%%%%%%%%%%%%%%%
%%%%%%%%%%%%%% BEGINNING %%%%%%%%%%%%%%
%%%%%%%%%%%%%%%%%%%%%%%%%%%%%%%%%%%

\begin{document}


%%%%%%%%%%%%%%%%%%%%%%
%%% Input your name, student number, 
%%% project and report details

\def\studentname{Philip Corr}
\def\projecttitle{ConvNets for iOS Gesture Recognition Applications}
\def\ucdstudentnumber{12318581}
\def\weeklyreportnumber{7}
\maketitle

%%%%%%%%%%%%%%%%%%%%%%ss
%%% First Section  

\section{Update}

\begin{itemize}

\item App nearly ready to role out. Few things to clear up in meeting but nothing major left.

\item Waiting to hear back about recording permission. Should hear back this week.

\item Did some research into sequence to sequence RNN's. Will implement as soon as app is finished.
	
\end{itemize}

\section{Items for discussion}

\begin{itemize}
\item Can skip a scene passing data through a segue?

background color to seperate from keyboard

bigger buttons on detailsentry

background of age field is age

two pictures for male/female

remove nationality and gender text

buttons on numberEntry page

dismiss keyboard when other button clicked

flexmonkey spline interpolation cubic smoother curves hermites


\item Fire segue from scene A to B when clicking button in segue C?

 Not possible - move scene by scene

\item Best way to initiate variable that doesn’t get a value until segue happens?

put all data in object... pass object along
It's fine to initialise as I am...

\item which finger?

\item How to improve smoothness of number tracing? - may answer next 2 questions also

\item Best way to draw two strokes? Two arrays? Will user use 3 strokes ever? assume 2 max?

options:
resample points so that always same size into NN, or use bitmap, or variable length?
Tangents?

Best:
Multistage algorithm that completes as soon as it is done... half way through etc.
State machine?
Different models trained?

Force may not be an awful lot of info...
Correlation between that and the speed etc. 

do both index and thumb

ask if left handed or right handed at the end

could do time from the start and indefinitely extract

protobuffers - google

create 28x28 bit images and use as test set on trained MNIST

also other way? - convert MNIST to strokes...

arc length and first derivative...

\item How to time strokes? start at beginning of first stroke and end at end of second?

item lack of force??

\end{itemize}

\section{Plan for Next Week}
In order of priority from top to bottom

\begin{itemize}
\item Further permission to record data

\item Get data stored in a reliable way

\item Make a timetable for the year, more detailed one until christmas?

\item Start into further deep learning resources e.g.~\cite{Ng-Coursera-2016, VincentVanhoucke-Udacity-2016, Nvidia-DL-Course-2016}. Tensorflow also.

\item Investigate paramaterisation of bitmaps.

\end{itemize}

\section{Meeting Notes}




%%%%%%%%%%%%%%%%%%%%%%
%%% Bibliography

\bibliography{report-biblio}{}
\bibliographystyle{IEEEtran}

%%%%%%%%%%%%%%%%%%%%%%%%%%%%%%%%%%%
%%%%%%%%%%%%%% END %%%%%%%%%%%%%%%%%%
%%%%%%%%%%%%%%%%%%%%%%%%%%%%%%%%%%%

\label{last_page}

 \end{document} 