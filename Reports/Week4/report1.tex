\documentclass[]{weekly-report}
\usepackage{hyperref}
 
%%%%%%%%%%%%%%%%%%%%%%%%%%%%%%%%%%%
%%%%%%%%%%%%%% BEGINNING %%%%%%%%%%%%%%
%%%%%%%%%%%%%%%%%%%%%%%%%%%%%%%%%%%

\begin{document}


%%%%%%%%%%%%%%%%%%%%%%
%%% Input your name, student number, 
%%% project and report details

\def\studentname{Philip Corr}
\def\projecttitle{ConvNets for iOS Gesture Recognition Applications}
\def\ucdstudentnumber{12318581}
\def\weeklyreportnumber{2}
\maketitle

%%%%%%%%%%%%%%%%%%%%%%ss
%%% First Section  

\section{Update}

\begin{itemize}

\item Set up Xcode and built calculator from stanford module~\cite{PaulHegarty2016}. Very good and Lecturer explains very clearly. I've gone through Basic syntax and MVC so far. This was done in the first 3 videos.

\item Finished Coursera videos. Went through variance and bias. Also went through problems that can arise and when certain solutions apply.
Highlighted some issues to be aware of and avoid.
	- More data isn't always the right answer
	- Cross validation set allows for more valid testing
	- When to add more features/neurons to the network
	
\item Set up Latex through Texmaker and Texlive

\item Signed risk assessment form
	
\end{itemize}

\section{Items for discussion}

\begin{itemize}

\item Would like to set up app asap and then do further state of the art research as I am collecting data. For this reason I suggest devoting as much of this week as needed to get app up and running and recording some sample data 

\item Is it ok to record meetings electronically in Latex and provide it as supplement to my lab book? What is log-book folder/documentation folder for? Doesn't it have to be hard copy? Will I use bib for ongoing references vs weekly bib for stuff I find during the week? Should I use reference manager?

\item Use a cross validation set? Implications on data set to be recorded?

\item Experiment with GPU? Do I need CUDA for this?

\item Discuss timetable? Maybe better to do this after I have a rough outline made?

\item Any advice on where to work when I graduate?

\item Is it possible to do a PHD in few years? What are the differences between doing one now and then?

\item Funding for PHD?

\item Working abroad vs. Ireland?

\end{itemize}

\section{Plan for Next Week}
In order of priority from top to bottom

\begin{itemize}
\item Get app up and running.

\item Make a timetable for the year, more detailed one until christmas?

\item Start into further deep learning resources e.g.~\cite{Ng-Coursera-2016, VincentVanhoucke-Udacity-2016, Nvidia-DL-Course-2016}.

\item Investigate paramaterisation of bitmaps.

\end{itemize}

\section{Meeting Notes}
canadian paper using RNNS, show attend and tell, theano and tensorflow

We will probably use tensorflow

More details on LeNet in next meeting - paper overview first + code snippets
training it and running it on the test data

iOS uses SQLite - seamless integation with cloud
Imageview - generate with photoshop and overlay?
GitHub - UIView that displays a grid - see what is available
Don't want grid Layout, want to display a grid - s,y and drawrec

serialisation class for JSON - one line of code - apple supported

stack buttons - another way to do grid layout

Look into force? Api exists or not? May only be exposed as a gesture?
UItouch api - CGFloat
Get from gesture recogniser
Possibly get Iphone6s?

Go through udacity stuff + softmax

Encrypt hard drive?

Need to build latex twice to allow dependencies to be resolved
IQ test here has to be done for America
Switzerland or Scotland
Do PHD now or never do one.
Matter of getting funding in UCD - Research proposals - SFI
Find out if funding is successful around May
deadline for SFI - IRSECT - programme for supporting PHDS
In US apply to grad schools, masters phases into PHD... or other way around...
Possibility for Qlik supporting PHD
Daily tasks, languages? 
NYU
Stanford 
Toronto
Areas that overlap with signal processing
What sub area in machine learning are you interested in? Genomics? Big data? Medical? Reinforcement learning
Google research - get contact from papers other places? Email and ask who to contact - one response in 10

%%%%%%%%%%%%%%%%%%%%%%
%%% Bibliography

\bibliography{report-biblio}{}
\bibliographystyle{IEEEtran}

%%%%%%%%%%%%%%%%%%%%%%%%%%%%%%%%%%%
%%%%%%%%%%%%%% END %%%%%%%%%%%%%%%%%%
%%%%%%%%%%%%%%%%%%%%%%%%%%%%%%%%%%%

\label{last_page}

 \end{document} 